
% The \section{} command formats and sets the title of this
% section. We'll deal with labels later.
\section{Introduction}
\label{sec:intro}

The 8-puzzle is one of a number of relatively simple problems, known as toy problems, which arise from logic tasks and are used to illustrate algorithm behavior, but do not describe real-world problems. The sliding-block family of puzzles are difficult to solve, with large state spaces, but have easily established representations and heuristics and have been historically used to evaluate search algorithms \cite{intractable}. \\

The 8$\times$8 puzzle has 181,400 reachable states \cite{aima}.  Previous analyses exploring larger puzzles have shown that solving the N$\times$N puzzle is NP-hard and thus computationally intractable \cite{intractable}.  The problem is therefore an appropriate target for the lower memory costs and autonomous execution of a local search algorithm. \\

In this paper, we describe the heuristics used by the A* algorithm, detail our experimental approach, including our random board generation method, and then present the results of the comparison in heuristics.

%In this section, you should introduce the reader to the problem you
%are attempting to solve. For example, for the first project: describe
%the $15$-puzzle, and why it's interesting as an A.I. problem. You
%should also cite and briefly describe other related papers that have
%tackled this problem in the past --- things that came up during the
%course of your research. In the AAAI style, citations look like
%\cite{aima} (see the comments in the source file \texttt{intro.tex} to
%see how this citation was produced). Conclude by summarizing how the
%remainder of the paper is organized. \\

% Citations: As you can see above, you create a citation by using the
% \cite{} command. Inside the braces, you provide a "key" that is
% uniue to the paper/book/resource you are citing. How do you
% associate a key with a specific paper? You do so in a separate bib
% file --- for this document, the bib file is called
% project1.bib. Open that file to continue reading...

% Note that merely hitting the "return" key will not start a new line
% in LaTeX. To break a line, you need to end it with \\. To begin a 
% new paragraph, end a line with \\, leave a blank
% line, and then start the next line (like in this example).

%Overall, the aim in this section is context-setting: what is the
%big-picture surrounding the problem you are tackling here?

